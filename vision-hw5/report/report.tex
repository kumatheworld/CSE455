\documentclass[12pt]{article}
\usepackage[margin=1in]{geometry}
\usepackage{enumitem}
\usepackage{caption}
\usepackage{subcaption}
\usepackage{graphicx}
\graphicspath{{figures/}}

\title{
  \vspace{-2cm}
  CSE 455 Homework 5 \\
  \author{Yoshihiro Kumazawa}
}

\begin{document}
\maketitle

\section{Installing PyTorch}
Done!

\section{Find the best network}
\subsection{Training a classifier using only one fully connected Layer}
See Figure~\ref{fig:2_1}. We can say that the the model successfully trained since the loss is decreasing throughout the training process and there is a healthy gap between the training accuracy and testing accuracy.
\begin{figure}
  \centering
  \begin{subfigure}{.5\textwidth}
    \centering
    \includegraphics[width=.8\linewidth]{accuracies_2_1.png}
    \caption{Accuracy}
    \label{fig:2_1_sub1}
  \end{subfigure}%
  \begin{subfigure}{.5\textwidth}
    \centering
    \includegraphics[width=.8\linewidth]{loss_2_1.png}
    \caption{Training loss}
    \label{fig:2_1_sub2}
  \end{subfigure}
  \caption{Training result of LazyNet}
  \label{fig:2_1}
\end{figure}

\subsection{Training a classifier using multiple fully connected Layers}
See Figure~\ref{fig:2_2}. The training is not successful because the testing accuracy plateaus whereas the training keeps increasing.
\begin{figure}
  \centering
  \begin{subfigure}{.5\textwidth}
    \centering
    \includegraphics[width=.8\linewidth]{accuracies_2_2.png}
    \caption{Accuracy}
    \label{fig:2_2_sub1}
  \end{subfigure}%
  \begin{subfigure}{.5\textwidth}
    \centering
    \includegraphics[width=.8\linewidth]{loss_2_2.png}
    \caption{Training loss}
    \label{fig:2_2_sub2}
  \end{subfigure}
  \caption{Training result of BoringNet}
  \label{fig:2_2}
\end{figure}

\subsubsection{Question}

\subsection{Training a classifier using convolutions}
\subsubsection{Question}

\section{How does learning rate work?}

\section{Data Augmentation}

\section{Change the loss function}

\end{document}
